\documentclass[tikz,margin=1mm]{standalone}


% settings for laser cutting
\tikzset{every path/.style={line width=0.035mm, draw=blue,text=black}}
\usepackage[outline]{contour}
\contourlength{0.035mm}
\newcommand{\textcontour}[1]{\contour{red}{\textcolor{white}{#1}}}


\usepackage{xfp} % for higher precision floating point math

% sans serif font
\renewcommand{\familydefault}{\sfdefault}
\usepackage{helvet}

\begin{document}

\newcommand{\n}{10}         % number of cuts
\newcommand{\base}{0.25}    % smallest cut
\newcommand{\spacing}{1.2}  % space between the cuts
\newcommand{\kerf}{0.005}   % kerf increment



\newcommand{\aaa}{2}        
\newcommand{\bbb}{1}  
\newcommand{\ccc}{-0.3}
\newcommand{\ddd}{-1.8}
\newcommand{\eee}{1}
\newcommand{\fff}{0.2}



\begin{tikzpicture}

    \def\points{}

    \xdef\points{\points (0,\aaa) --}  

    % the cuts of varying width
    \foreach \i in {0,1,...,\numexpr\n-1} {
        \pgfmathsetmacro\x{(1+\i)*\spacing}  
        \pgfmathsetmacro\y{\aaa}   
        \xdef\points{\points (\x,\y) --}     
        \pgfmathsetmacro\x{(1+\i)*\spacing}  
        \pgfmathsetmacro\y{\bbb}  
        \xdef\points{\points (\x,\y) --}  
        \pgfmathsetmacro\x{(1+\i)*\spacing+\base+\i*\kerf}  
        \pgfmathsetmacro\y{\bbb}  
        \xdef\points{\points (\x,\y) --}  
        \pgfmathsetmacro\x{(1+\i)*\spacing+\base+\i*\kerf}  
        \pgfmathsetmacro\y{\aaa}  
        \xdef\points{\points (\x,\y) --}  
    }

    \pgfmathsetmacro\x{(\n+1)*\spacing}  
    \pgfmathsetmacro\y{\aaa}  
    \xdef\points{\points (\x,\y) --}  

    \pgfmathsetmacro\x{(\n+1)*\spacing}  
    \pgfmathsetmacro\y{0}  
    \xdef\points{\points (\x,\y) --}  

    % the pieces for the kerf measurement
    \foreach \i in {\numexpr\n-1,\numexpr\n-2,...,0} {

        \pgfmathsetmacro\x{(1+\i)*\spacing}  
        \pgfmathsetmacro\y{0}   
        \xdef\points{\points (\x,\y) --}  

        \pgfmathsetmacro\x{(1+\i)*\spacing}  
        \pgfmathsetmacro\y{\ccc}   
        \xdef\points{\points (\x,\y) --}  

        \pgfmathsetmacro\x{(1+\i)*\spacing + (\eee-\fff)/2}  
        \pgfmathsetmacro\y{\ccc}   
        \xdef\points{\points (\x,\y) --}  

        \pgfmathsetmacro\x{(1+\i)*\spacing + (\eee-\fff)/2}  
        \pgfmathsetmacro\y{\ddd}   
        \xdef\points{\points (\x,\y) --}  

        \pgfmathsetmacro\x{(1+\i)*\spacing + (\eee-\fff)/2-\eee}  
        \pgfmathsetmacro\y{\ddd}   
        \xdef\points{\points (\x,\y) --}

        \pgfmathsetmacro\x{(1+\i)*\spacing + (\eee-\fff)/2-\eee}  
        \pgfmathsetmacro\y{\ccc}   
        \xdef\points{\points (\x,\y) --}

        \pgfmathsetmacro\x{(1+\i)*\spacing -\fff}  
        \pgfmathsetmacro\y{\ccc}   
        \xdef\points{\points (\x,\y) --}

        \pgfmathsetmacro\x{(1+\i)*\spacing -\fff}  
        \pgfmathsetmacro\y{0}   
        \xdef\points{\points (\x,\y) --}
    }

    \xdef\points{\points (0,0)}  


    % draw!
    \draw[] \points -- cycle;

    % annotate the cuts
    \foreach \i in {0,1,...,\numexpr\n-1} {
        \node at (\fpeval{\base/2+(1+\i)*\spacing+\i*\kerf/2},0.65) {\textcontour{\textbf{\fpeval{round(100*(10*(\base+\i*\kerf)))/100}}}}; 
    }

\end{tikzpicture}

\end{document}