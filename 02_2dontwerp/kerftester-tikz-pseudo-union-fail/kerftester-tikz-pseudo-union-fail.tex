\documentclass[tikz,margin=5mm]{standalone}
%\usetikzlibrary{...}% tikz package already loaded by 'tikz' option
 
% settings for laser cutting
\tikzset{every path/.style={line width=0.035mm, draw=red,text=black}}
\usepackage[outline]{contour}
 
\newcommand{\tikzlinewidth}{0.035mm}
 
\contourlength{\tikzlinewidth}
\newcommand{\textcontour}[1]{\contour{red}{\textcolor{white}{#1}}}
 
 
\usepackage{xfp} % for higher precision floating point math
 
% sans serif font
\renewcommand{\familydefault}{\sfdefault}
\usepackage{helvet}
 
\begin{document}
 
\newcommand{\n}{10}         % number of cuts
\newcommand{\base}{0.26}    % smallest cut
\newcommand{\spacing}{1.2}  % space between the cuts
\newcommand{\kerf}{0.005}   % kerf increment
 
\begin{tikzpicture}
 
    \draw[line width=2*\tikzlinewidth,postaction={white,fill}]
    (0,0) rectangle ++(\fpeval{\spacing*(\n+1)},1)
    (0,0.9) rectangle ++(\spacing,1.1)
 
    \foreach \i in {0,1,...,\numexpr\n-1} {
        (\fpeval{\spacing + \base + \i*\spacing+\i*\kerf},0.9) rectangle ++(\fpeval{\spacing-\base-\i*\kerf},1.1) 
    }
 
    \foreach \i in {0,1,...,\numexpr\n-1} {
        (\fpeval{\spacing/2 + \i*\spacing},-1.8) rectangle ++(1,1.5)
        (\fpeval{\spacing + \i*\spacing -0.2},-1) rectangle ++(0.2,1.2)     
    };
 
    \foreach \i in {0,1,...,\numexpr\n-1} {
        \node at (\fpeval{\base/2+(1+\i)*\spacing+\i*\kerf/2},0.65) {\textbf{\fpeval{round(100*(10*(\base+\i*\kerf)))/100}}}; 
    }
 
\end{tikzpicture}
 
\end{document}